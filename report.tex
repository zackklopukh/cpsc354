\documentclass{article}

\usepackage{tikz} 
\usetikzlibrary{automata, positioning, arrows} 

\usepackage{amsthm}
\usepackage{amsfonts}
\usepackage{amsmath}
\usepackage{amssymb}
\usepackage{fullpage}
\usepackage{color}
\usepackage{parskip}
\usepackage{hyperref}
  \hypersetup{
    colorlinks = true,
    urlcolor = blue,       % color of external links using \href
    linkcolor= blue,       % color of internal links 
    citecolor= blue,       % color of links to bibliography
    filecolor= blue,        % color of file links
    }
    
\usepackage{listings}

\definecolor{dkgreen}{rgb}{0,0.6,0}
\definecolor{gray}{rgb}{0.5,0.5,0.5}
\definecolor{mauve}{rgb}{0.58,0,0.82}

\lstset{frame=tb,
  language=haskell,
  aboveskip=3mm,
  belowskip=3mm,
  showstringspaces=false,
  columns=flexible,
  basicstyle={\small\ttfamily},
  numbers=none,
  numberstyle=\tiny\color{gray},
  keywordstyle=\color{blue},
  commentstyle=\color{dkgreen},
  stringstyle=\color{mauve},
  breaklines=true,
  breakatwhitespace=true,
  tabsize=3
}

\newtheoremstyle{theorem}
  {\topsep}   % ABOVESPACE
  {\topsep}   % BELOWSPACE
  {\itshape\/}  % BODYFONT
  {0pt}       % INDENT (empty value is the same as 0pt)
  {\bfseries} % HEADFONT
  {.}         % HEADPUNCT
  {5pt plus 1pt minus 1pt} % HEADSPACE
  {}          % CUSTOM-HEAD-SPEC
\theoremstyle{theorem} 
   \newtheorem{theorem}{Theorem}[section]
   \newtheorem{corollary}[theorem]{Corollary}
   \newtheorem{lemma}[theorem]{Lemma}
   \newtheorem{proposition}[theorem]{Proposition}
\theoremstyle{definition}
   \newtheorem{definition}[theorem]{Definition}
   \newtheorem{example}[theorem]{Example}
\theoremstyle{remark}    
  \newtheorem{remark}[theorem]{Remark}

\title{CPSC-354 Report}
\author{Zack Klopukh \\ Chapman University}

\date{\today} 

\begin{document}

\maketitle

\begin{abstract}
(Delete and Replace:) You can safely delete and replace the explanations in this file as they will remain available on the course website. For example, you should replace this abstract with your own. The abstract should be a short summary of the report. It should be written in a way that makes it possible to understand the purpose of the report without reading it.  
\end{abstract}

\setcounter{tocdepth}{3}
\tableofcontents

\section{Introduction}\label{intro}

(Delete and Replace): This report will document your learning throughout the course. It will be a collection of your notes, homework solutions, and critical reflections on the content of the course. Something in between a semester-long take home exam and your own lecture notes.\footnote{One purpose of giving the report the form of lecture notes is that self-explanation is a technique proven to help with learning, see Chapter 6 of Craig Barton, How I Wish I'd Taught Maths, and references therein. In fact, the report can lead you from self-explanation (which is what you do for the weekly deadline) to explaining to others (which is what you do for the final submission). Another purpose is to help those of you who want to go on to graduate school to develop some basic writing skills. A report that you could proudly add to your application to graduate school (or a job application in industry) would give you full points.}



The full report is due at the end of the finals week. It will be graded according to the following guidelines.

Grading  guidelines (see also below):
\begin{itemize}
\item Is typesetting and layout professional? 
\item Is the technical content, in particular the homework, correct?
\item Did the student find interesting references~\cite{bla} and cites them throughout the report?
\item Do the notes reflect understanding and critical thinking?
\item Does the report contain material related to but going beyond what we do in class?
\item Are the questions interesting?
\end{itemize}

Do not change the template (fontsize, width of margin, spacing of lines, etc) without asking your first.

\section{Week by Week}\label{homework}

\subsection{Week 1}

\subsubsection*{Notes}

Using proofs is important for reasoning on how computers can operate.

\subsubsection*{Homework}

5) Goal: $a+(b+0)+(c+0)=a+b+c$

 \verb{rw [add_zero b, add_zero c]
\\
$a+b+c=a+b+c$ \\
rfl
\\
\verb{ Here we use  add_zero to rewrite and remove the extra zeros so that both sides are equal by reflexivity.
\\ \\
6) Goal: $a+(b+0)+(c+0)=a+b+c$

 \verb{rw [add_zero b, add_zero c]
 \\
 $a+b+c=a+b+c$ \\
 rfl
A nearly identical solution as number 5. \\

7) Goal: $succ n = succ(n+0)$
\verb{ rw [add_zero]
\\
rfl \\
\verb{ Here we used add_zero to rewrite n+0 to n which was all it took to show equality through reflexivity.
\\ \\
8) Goal: $2+2=4$ \\
\verb { rw [four_eq_succ_three]
\\
2 + 2 = succ 3 \\
\verb. rw [three_eq_succ_two]
\\
2 + 2 = succ (succ 2) \\
\verb. rw [two_eq_succ_one]
\\
succ 1 + succ 1 = succ (succ (succ 1)) \\
\verb/ rw [one_eq_succ_zero]
\\ succ (succ 0) + succ (succ 0) = succ (succ (succ (succ 0))) \\
\verb/ rw [add_succ]
\\ succ (succ (succ 0) + succ 0) = succ (succ (succ (succ 0))) \\
\verb / rw [add_succ]
\\ succ (succ (succ (succ 0) + 0)) = succ (succ (succ (succ 0))) \\
\verb/ rw [add_zero]
\\ succ (succ (succ (succ 0))) = succ (succ (succ (succ 0)))
\\
\verb / rfl
\\
This is the longest proof in the homework and involves multiple rewrittings to show. The process I took was turning both 2's and the 4 into their forms as successors of 0. From that point I added successors to the left side and by doing so I rewrite the two separate 2's into one large successor. Finally by using add zero, I removed one of the two left terms completely making it a single term on the left and right, both representative of 4.



% Make sure that this section can be read without referring back to the homework question. Introduce the question/problem and repeat it in your own words. Make sure to typset your homework in a way that makes it clear what  the question and what the answer is. Present it as a worked example would be presented in a textbook. 

% Also explain what you learn from the homework. Each homework was carefully drafted to bring home a particular teaching point. Make sure to explain what this point is. Relate it to the big questions mentioned above. 

%In case you want to draw automata in Latex, you can use the tikz %package. Here is an example of a simple automaton:
%
%\begin{tikzpicture}[shorten >=1pt,node distance=2cm,on grid,auto] 
%  \node[state] (q_1)   {$q_1$}; 
%  \node[state] (q_2) [above right=of q_1] {$q_2$}; 
%  \node[state] (q_3) [below right=of q_2] {$q_3$}; 
%   \path[->] 
%   (q_1) edge  node {0} (q_2)
%         edge  node [swap] {1} (q_3)
%   (q_2) edge  node  {1} (q_3)
%         edge [loop above] node {0} ()
%   (q_3) edge [loop below] node {0,1} ();
%\end{tikzpicture}
%
%By the way, GPT-4 is quite good at outputting tikz code.

\subsubsection*{Comments and Questions}

This introduction world taught me how precise proofs need to be. I am beginning to understand how important it is to a machine that proofs are completely concrete, to the point where 2+2=4 becomes a tricky problem. As I go on to take these rewrites for granted, I will with the understanding that under the hood the computer does not go off intuition, but precise and exact steps. \\\\
 Question of the week:
How can systems of algorithmic reasoning as shown in the tutorial world, based in discrete mathematics, be encoded in modern programming languages?
%I expect you to read the lecture notes. 

\subsection{Week 2}

(Delete:) Week 2 (and all the other weeks) should follow the same pattern as Week 1. Even if there is a week without homework, notes and comments (see above) are still expected.

\subsection{\ldots}

\ldots

\section{Lessons from the Assignments}

(Delete and Replace): Write three pages about your individual contributions to the project.

On 3 pages you describe lessons you learned from the project. Be as technical and detailed as possible. Particularly valuable are \emph{interesting} examples where you connect concrete technical details with \emph{interesting} general observations or where the theory discussed in the lectures helped with the design or implementation of the project.


Write this section during the semester. This is approximately a quarter of apage per week and the material should come from the work you do anyway. Just keep your eyes open for interesting lessons.

Make sure that you use \LaTeX{} to structure your writing (eg by using subsections).

\section{Conclusion}\label{conclusion}

(Delete and Replace): (approx 400 words) A critical reflection on the content of the course. Step back from the technical details. How does the course fit into the wider world of software engineering? What did you find most interesting or useful? What improvements would you suggest?

\begin{thebibliography}{99}
\bibitem[BLA]{bla} Author, \href{https://en.wikipedia.org/wiki/LaTeX}{Title}, Publisher, Year.
\end{thebibliography}

\end{document}